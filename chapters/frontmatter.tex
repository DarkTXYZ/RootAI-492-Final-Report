\maketitle
\makesignature

\ifproject
\begin{abstractTH}
เป้าหมายของโครงการนี้คือ การพัฒนาระบบปัญญาประดิษฐ์ที่สามารถเล่นเกมกระดาน ``Root'' ได้อย่างชาญฉลาดและมีประสิทธิภาพ โดยระบบจะรับ state ปัจจุบันของเกม และทำการตัดสินใจเลือกทำ action ต่างๆในระหว่างการเล่น

Root เป็นเกมแนวอสมมาตร (asymmetric) และ  ที่สามารถสังเกตได้บางส่วน (partially observable) โดยผู้เล่นแต่ละคนมีวิธีการเล่นเกมและเป้าหมายที่แตกต่างกัน ตามเผ่า (faction) ที่ตนเองได้เล่น ซึ่งในเกมกระดาน Root ประกอบไปด้วยหลายเผ่ามากมาย แต่ในโครงการนี้เราจะสนใจที่ 2 เผ่าหลัก ได้แก่เผ่า ``Marquise'' de Cat และเผ่า the ``Eyrie'' Dynasties

อัลกอริทึมหลักที่เราจะใช้ในการสร้างปัญญาประดิษฐ์คือ Monte Carlo Tree Search (MCTS) ซึ่งเป็นการหาความน่าจะเป็นในการชนะของแต่ละ action ที่ทำได้โดยการจำลองการเล่นด้วย action นั้นซ้ำหลายๆรอบ จากนั้นเลือก action ที่ทำให้ความน่าจะเป็นในการชนะสูงที่สุด

อัลกอริทึม MCTS มี parameters หลายตัว นั่นทำให้มี variant ของ MCTS อยู่หลาย variant เราจะนำ MCTS แต่ละ variant มาแข่งเพื่อหา variant ของ MCTS ที่ดีที่สุดสำหรับแต่ละเผ่า

% เขียนบทคัดย่อของโครงงานที่นี่

% การเขียนรายงานเป็นส่วนหนึ่งของการทำโครงงานวิศวกรรมคอมพิวเตอร์
% เพื่อทบทวนทฤษฎีที่เกี่ยวข้อง อธิบายขั้นตอนวิธีแก้ปัญหาเชิงวิศวกรรม และวิเคราะห์และสรุปผลการทดลองอุปกรณ์และระบบต่างๆ
% \enskip อย่างไรก็ดี การสร้างรูปเล่มรายงานให้ถูกรูปแบบนั้นเป็นขั้นตอนที่ยุ่งยาก
% แม้ว่าจะมีต้นแบบสำหรับใช้ในโปรแกรม Microsoft Word แล้วก็ตาม
% แต่นักศึกษาส่วนใหญ่ยังคงค้นพบว่าการใช้งานมีความซับซ้อน และเกิดความผิดพลาดในการจัดรูปแบบ กำหนดเลขหัวข้อ และสร้างสารบัญอยู่
% \enskip ภาควิชาวิศวกรรมคอมพิวเตอร์จึงได้จัดทำต้นแบบรูปเล่มรายงานโดยใช้ระบบจัดเตรียมเอกสาร
% \LaTeX{} เพื่อช่วยให้นักศึกษาเขียนรายงานได้อย่างสะดวกและรวดเร็วมากยิ่งขึ้น
\end{abstractTH}

\begin{abstract}
Our goal is to develop an artificial intelligence (AI) system capable of playing the board game ``Root''. Our objective is to create an intelligent agent that can take game state as input and make intelligent decisions in each phase.


Root is a partially observable asymmetric game where each player has different game mechanics and goals. There are many factions within Root, each having its own way to achieve victory conditions. We will focus on two base factions: ``Marquise'' de Cat and the ``Eyrie'' Dynasties.

Our main algorithm for implementing the AI is Monte Carlo Tree Search (MCTS). MCTS algorithm maximizes the probability of winning by simulating many rollouts and choosing the best decision.

There are multiple parameters for the MCTS algorithm resulting in multiple variants of MCTS algorithms. We will pitch different variants of MCTS together with aim to find the best MCTS variant for each faction.

% To build the AI, we will construct multiple AI agents and have them play against one another so they can both improve themselves. Each agent will have different versions built from different algorithms, including Monte Carlo Tree Search, Neural Network Reinforcement Learning, random, weighted-random, hard-coded, human player, etc.

% We aim to develop an artificial intelligence (AI) system capable of playing the board game "Root." Our objective is to create an intelligent agent that can take game state as input, make decisions in each phase, and compete against the AI in the digital adaptation video game of the board game Root.

% Root is a partially observable asymmetric game where each player has different game mechanics and goals. This can be applied to real-world problems where different parties have different goals but are in the same area, so they have to compete to take hold of the limited resources needed to complete their goals.

% To build the AI, we will have an agent play against another agent so they can both learn. Each agent will have different versions built from different algorithms, including Monte Carlo Tree Search, Neural Network Reinforcement Learning, random, weighted-random, hard-coded, human player, etc.
% ------
% The abstract would be placed here. It usually does not exceed 350 words
% long (not counting the heading), and must not take up more than one (1) page
% (even if fewer than 350 words long).

% Make sure your abstract sits inside the \texttt{abstract} environment.
\end{abstract}

\iffalse
\begin{dedication}
This document is dedicated to all Chiang Mai University students.

Dedication page is optional.
\end{dedication}
\fi % \iffalse

\begin{acknowledgments}
We would like to express our deep appreciation for Asst. Prof. Kasemsit Teeyapan, Ph.D. for his exceptional dedication, expertise, and detailed guidance throughout the project, which served as a constant source of inspiration and helped ensure its successful completion.
% Your acknowledgments go here. Make sure it sits inside the
% \texttt{acknowledgment} environment.

\acksign{2024}{02}{18}
\end{acknowledgments}%
\fi % \ifproject

\contentspage

\ifproject
\figurelistpage

\tablelistpage
\fi % \ifproject

% \abbrlist % this page is optional

% \symlist % this page is optional

% \preface % this section is optional
