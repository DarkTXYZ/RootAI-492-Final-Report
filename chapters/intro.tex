\chapter{\ifenglish Introduction\else บทนำ\fi}

\section{\ifenglish Project rationale\else ที่มาของโครงงาน\fi}
Artificial intelligence (AI) has revolutionized the world of video games by giving players friends to play with or foes to play against. The digital adaptation video game of the \RootB{} is no exception. But there are problems with the AIs. There are multiple reports by \RootV{} players that the AIs are no longer challenging to play against once you start to know how to play your faction or that the AIs make awful decisions that no humans would ever make in a similar situation.

This project focuses on the recommended scenario in a two-player game of \Marquise{} faction versus \Eyrie{} faction since they are both empire factions, i.e., factions with many buildings and warriors on the board, thus making this scenario have a lot of interactions.

%%%%%
%
% Maybe we should move this paragraph vvv to background chapter 
%
\Marquise{} requires players to balance building types to be able to continuously grow and earn points, while \Eyrie{} requires players to carefully build and follow \textit{the Decree}, which can get very powerful, but failing to follow it will result in a setback in the form of losing points. The relevant rules of Root will be explained in this document.
%
%%%%%

This project aims to create an artificial intelligence (AI) system capable of playing the board game ``Root'' that is more intelligent than the AIs in the \RootV{}. The agents will be built using different methods, including Monte Carlo tree search, neural network reinforcement learning, random, human player, etc. Full details on each AI agent implementation method and their concepts are explained in this document.

% Root features a variety of factions, each with its unique gameplay mechanics and goals. 

% We will implement our own version of Root video game in Python which allows us to freely control the environment and integrate with other libraries for developing game AIs. Then  

% Our objective is to create an intelligent agent that can take game state as input, make decisions in each phase, and compete against the AI in the digital adaptation video game of the board game Root.

% One of these factions, the Eyrie Dynasties, presents a particularly intricate challenge when it comes to AI gameplay due to its unique mechanics that allows it to lose victory points. This project aims to address the existing limitations in AI within the Root video game, focusing on enhancing the experience for the Eyrie Dynasties faction.


\section{\ifenglish Objectives\else วัตถุประสงค์ของโครงงาน\fi}
\begin{itemize}
    \item To create an artificial intelligence (AI) that can play the board game ``Root'' and win against the AIs in the \RootV{}.
    % \item To create a version of Root video game with integrated AI agent training lifecycle which also allows human players to play against the AI agents.
\end{itemize}

\section{\ifenglish Project scope\else ขอบเขตของโครงงาน\fi}
% TODO: should i use AI agent training wrapper? training framework? training lifecycle?
The project is a software consisting of two main components: 1.) \textit{\RootOurs{}}, a minimal version of \RootV{}. 2.) \textit{\RootAI{}}, an AI agent training framework for \RootOurs{}. The scope of the project are as follows:
\begin{enumerate}
    \item Implementation of \RootOurs{}.
    \item Implementation of \RootAI{}.
    \item Exclude direct integration with \RootV{}, i.e., no pulling data from a running \RootV{}, and no injecting decisions into a running \RootV{} via code.
    \item Exclude direct integration with \RootB{}, i.e., no computer vision will be used to detect the state of the board game.
\end{enumerate}

\RootOurs{} supports the rules of Root [\ref{brief-rules-of-root}] in the scenario of \Marquise{} versus \Eyrie{}. Game mechanics not used by these two factions are not included. The scope of \RootOurs{} are as follows:
\begin{enumerate}
    \item Only \Marquise{} and \Eyrie{} factions.
    \item Only ``Woodland'' map.
    \item Exclude \textit{forests} area.
    \item Exclude building type \textit{ruins}.
    % \item \Marquise{} starts with their \textit{the Keep} token in the bottom right \textit{clearing}.
    % \item \Marquise{} starts with their three starting buildings pre-placed.
    % \item \Eyrie{} starts with a \textit{roost} in the top left clearing.
    % \item \Eyrie{} starts with the \textit{charismatic} leader.
\end{enumerate}

\RootAI{} has the following features:
\begin{enumerate}
    \item Importing and exporting of \RootOurs{}'s game state.
    \item Exporting of \RootOurs{}'s log.
    \item Interface for human interaction to an instance of \RootOurs{}.
    \item Training and running AI agents built using the following methods:
    \begin{itemize}
        \item Random decision
        \item Monte Carlo tree search (MCTS)
        \item Reinforcement learning neural network
    \end{itemize}
\end{enumerate}

% \subsection{\ifenglish Hardware scope\else ขอบเขตด้านฮาร์ดแวร์\fi}

% \subsection{\ifenglish Software scope\else ขอบเขตด้านซอฟต์แวร์\fi}

\section{\ifenglish Expected outcomes\else ประโยชน์ที่ได้รับ\fi}
\begin{itemize}
    \item The resulting AIs are able to win against \RootV{}'s corresponding opponent faction AI, i.e.:
    \begin{itemize}
        \item Resulting \Marquise{} AI have higher win rate against \RootV{}'s \Eyrie{} AI.
        \item Resulting \Eyrie{} AI have higher win rate against \RootV{}'s \Marquise{} AI.
    \item Root players have a more challenging opponent when playing by themselves.
    \end{itemize}
\end{itemize}

\section{\ifenglish Technology and tools\else เทคโนโลยีและเครื่องมือที่ใช้\fi}
Python is chosen as the language for this project's implementation due to its flexibility of integrating with AI related libraries such as PyTorch and Tensorflow. 

% \subsection{\ifenglish Hardware technology\else เทคโนโลยีด้านฮาร์ดแวร์\fi}
% \textit{Maybe add the gpu use and cloud use in the future}

\subsection{\ifenglish Software technology\else เทคโนโลยีด้านซอฟต์แวร์\fi}
%% TODO: add citations to each item's docs
% there are more
\textbf{Programming Languages:}
\begin{itemize}
    \item Python
\end{itemize}
% there are more
\textbf{Important Libraries:}
\begin{itemize}
    \item Pygame (UI) 
    \item PyTorch (RLNN)
    \item TensorFlow (RLNN)
\end{itemize}

\section{\ifenglish Project plan\else แผนการดำเนินงาน\fi}

\begin{plan}{6}{2023}{3}{2024}
    \planitem{6}{2023}{1}{2024}{\RootOurs \ implementation}
    \planitem{11}{2023}{2}{2024}{\RootAI \ implementation}
    \planitem{10}{2023}{11}{2023}{Agent:\textbf{Random}: Implementation}
    \planitem{10}{2023}{11}{2023}{Agent:\textbf{Random}: Simulation}
    \planitem{10}{2023}{11}{2023}{Agent:\textbf{Random}: Analysis and Documentation}
    \planitem{11}{2023}{2}{2024}{Agent:\textbf{MCTS}: Implementation}
    \planitem{2}{2024}{2}{2024}{Agent:\textbf{MCTS}: Simulation}
    \planitem{2}{2024}{2}{2024}{Agent:\textbf{MCTS}: Analysis and Documentation}
    \planitem{2}{2024}{3}{2024}{Document Finalization}
\end{plan}

\section{\ifenglish Roles and responsibilities\else บทบาทและความรับผิดชอบ\fi}
In \RootOurs{}, base classes and faction-neutral functions are divided equally among the two of us. Pawaret is responsible for \Marquise{}-related functionalities. Baangkok is responsible for \Eyrie{}-related functionalities.

In \RootAI{}, each step is divided equally, starting from method research to implementation to testing. This allows us to continuously exchange our knowledge about the current AI agent implementation method, allowing us to create and improve each AI agent to have good performance.

\section{\ifenglish%
Impacts of this project on society, health, safety, legal, and cultural issues
\else%
ผลกระทบด้านสังคม สุขภาพ ความปลอดภัย กฎหมาย และวัฒนธรรม
\fi}
This project can work as a foundation for future research on reinforcement learning, asymmetrical games, game AI, and other related topics.
