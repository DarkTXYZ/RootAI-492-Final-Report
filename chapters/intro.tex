\chapter{\ifenglish Introduction\else บทนำ\fi}

\section{\ifenglish Project rationale\else ที่มาของโครงงาน\fi}
Artificial intelligence (AI) has revolutionized the world of video games, giving players friends to play with or foes to play against. The digital adaptation video game of the \RootB{} is no exception. But there are problems with the AIs in . There are multiple reports by \RootV{} players that the AIs are no longer challenging to play against once you start to know how to play your faction or that the AIs make awful decisions that no humans would ever make in a similar situation.

This project focuses on the recommended scenario in a two-player game of \Marquise{} faction versus \Eyrie{} faction since they are both empire factions, i.e., factions with many buildings and warriors on the board, thus making this scenario have a lot of interactions.

%%%%%
%
% Maybe we should move this paragraph vvv to background chapter 
%
\Marquise{} requires players to balance building types to be able to continuously grow and earn points, while \Eyrie{} requires players to carefully build and follow \textit{the Decree}, which can get very powerful, but failing to follow it will result in a setback in the form of losing points. The relevant rules of Root will be explained in this document.
%
%%%%%

This project aims to create an artificial intelligence (AI) system capable of playing the board game "Root" that is more intelligent than the AIs in the \RootV{}. The agents will be built using different methods, including Monte Carlo tree search, neural network reinforcement learning, random, human player, etc. Full details on each AI agent implementation method and their concepts are explained in this document.

% Root features a variety of factions, each with its unique gameplay mechanics and goals. 

% We will implement our own version of Root video game in Python which allows us to freely control the environment and integrate with other libraries for developing game AIs. Then  

% Our objective is to create an intelligent agent that can take game state as input, make decisions in each phase, and compete against the AI in the digital adaptation video game of the board game Root.

% One of these factions, the Eyrie Dynasties, presents a particularly intricate challenge when it comes to AI gameplay due to its unique mechanics that allows it to lose victory points. This project aims to address the existing limitations in AI within the Root video game, focusing on enhancing the experience for the Eyrie Dynasties faction.


\section{\ifenglish Objectives\else วัตถุประสงค์ของโครงงาน\fi}
\begin{itemize}
    \item To create an artificial intelligence (AI) that can play the board game "Root" and win against the AIs in the \RootV{}.
    % \item To create a version of Root video game with integrated AI agent training lifecycle which also allows human players to play against the AI agents.
\end{itemize}

\section{\ifenglish Project scope\else ขอบเขตของโครงงาน\fi}

The project is a software consisting of two main component: 1.) \textit{\RootOurs{}}, a minimal version of \RootV{}. 2.) \textit{\RootAI{}}, the AI agent training lifecycle for \RootOurs{}. The scope of the project are as follows:
\begin{enumerate}
    \item Exclude direct integration with \RootV{}, i.e., no pulling data via code from a running video game, and no injecting decision via code to a running video game.
    \item Exclude direct integration with \RootB{}, i.e., no computer vision will be used to detect the state of the board game, and no mechanical arm will be used to move board game pieces.
\end{enumerate}

\RootOurs{} must support the rules of Root [\ref{brief-rules-of-root}] in the scenario of \Marquise{} versus \Eyrie{}. Game mechanics not used by these two factions are not included. The scope of \RootOurs{} are as follows:
\begin{enumerate}
    \item Only \Marquise{} and \Eyrie{} factions.
    \item Only "Woodland" map.
    \item Exclude \textit{forests} area.
    \item Exclude building type \textit{ruins}.
    \item \Marquise{} starts with their \textit{the Keep} token in the bottom right \textit{clearing}.
    \item \Marquise{} starts with their three starting buildings pre-placed.
    \item \Eyrie{} starts with a \textit{roost} in the top left clearing.
    \item \Eyrie{} starts with the \textit{charismatic} leader.
\end{enumerate}

\subsection{\ifenglish Hardware scope\else ขอบเขตด้านฮาร์ดแวร์\fi}

\subsection{\ifenglish Software scope\else ขอบเขตด้านซอฟต์แวร์\fi}

\section{\ifenglish Expected outcomes\else ประโยชน์ที่ได้รับ\fi}

\section{\ifenglish Technology and tools\else เทคโนโลยีและเครื่องมือที่ใช้\fi}

\subsection{\ifenglish Hardware technology\else เทคโนโลยีด้านฮาร์ดแวร์\fi}

\subsection{\ifenglish Software technology\else เทคโนโลยีด้านซอฟต์แวร์\fi}

\section{\ifenglish Project plan\else แผนการดำเนินงาน\fi}

\begin{plan}{6}{2020}{2}{2021}
    \planitem{7}{2020}{8}{2020}{ศึกษาค้นคว้า}
    \planitem{8}{2020}{1}{2021}{ชิล}
    \planitem{2}{2021}{2}{2021}{เผา}
    \planitem{12}{2019}{1}{2022}{ทดสอบ}
\end{plan}

\section{\ifenglish Roles and responsibilities\else บทบาทและความรับผิดชอบ\fi}
อธิบายว่าในการทำงาน นศ. มีการกำหนดบทบาทและแบ่งหน้าที่งานอย่างไรในการทำงาน จำเป็นต้องใช้ความรู้ใดในการทำงานบ้าง

\section{\ifenglish%
Impacts of this project on society, health, safety, legal, and cultural issues
\else%
ผลกระทบด้านสังคม สุขภาพ ความปลอดภัย กฎหมาย และวัฒนธรรม
\fi}

แนวทางและโยชน์ในการประยุกต์ใช้งานโครงงานกับงานในด้านอื่นๆ รวมถึงผลกระทบในด้านสังคมและสิ่งแวดล้อมจากการใช้ความรู้ทางวิศวกรรมที่ได้
